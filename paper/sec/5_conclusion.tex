\section{Conclusion}

%We have presented an efficient and effective approach for general video classification. Our key findings highlight that:

The paper sets out an efficient and effective approach for general video classification.

The following key findings are highlighted:

%\begin{itemize}
    %\item The proposed method significantly outperforms baselines in accuracy with a drastically reduced training time.
    %\item Integrating multimodal embeddings, particularly metadata embeddings derived from textual information, substantially enhances model performance.
    %\item Our novel data augmentation strategy applied in the embedding space is effective in mitigating overfitting and improving generalization.
    %\item A 4-layer MLP architecture provided the best trade-off between performance and complexity for our task.
%\end{itemize}

\begin{itemize}
    \item The proposed method demonstrates a substantial improvement in accuracy, accompanied by a significant reduction in training time.
    \item The integration of multimodal embeddings, notably metadata embeddings derived from textual information, has resulted in a substantial enhancement of model performance.
    \item The novel data augmentation strategy applied in the embedding space has been proved to be effective in mitigating overfitting and enhancing model generalization capabilities.
    \item The 4-layer MLP architecture was determined to provide the optimal trade-off between performance and complexity for the task at hand.
\end{itemize}

%This work demonstrates the potential of using carefully chosen pre-trained embeddings and systematic ablation studies to design lightweight yet powerful video classification models.

%While our method outperformed all baseline approaches on the evaluated tasks, it was only validated on a small subset of YouTube-8M. For a more comprehensive assessment and future work, it should be applied to larger-scale datasets such as Kinetics or the full YouTube-8M dataset.
% mention about limitations

This work demonstrates the potential of using carefully chosen pre-trained embeddings and systematic ablation studies to design a lightweight, yet powerful, video classification models. 

While the proposed methodology outperformed all baseline approaches on the evaluated tasks, it should be noted that validation was undertaken on a small subset of YouTube-8M.
For a more comprehensive assessment and future work, it should be applied to larger-scale datasets such as Kinetics or the original YouTube-8M dataset.

\newpage